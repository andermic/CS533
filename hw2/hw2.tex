\documentclass{article}
\usepackage[pdftex]{graphicx}
\usepackage{amsmath}
\usepackage{verbatim}
\author{Michael Anderson}
\title{Homework Set 2}
\begin{document}
\maketitle
\center{CS533}
\center{Prof. Fern}\\
\flushleft
\newpage

\begin{enumerate}
\item[\textbf{1.}]

\begin{enumerate}
\item[a)]
Yes, because if the goal is reachable in some time $T$ where 
$0 \le T \le T_{max}$, then $g^T$ and consequently 
$g^0 \vee ... \vee g^T \vee ... \vee g^{T_{max}}$ will be satisfiable
and the SATPLAN algorithm will return some plan.

\item[b)]
Yes. Suppose some goal is reachable at some time $t$ where $t \le T_{max}$.
Then all of the potential actions taken after $t$ will be arbitrary and all of
their possible combinations could be enumerated as solutions by the SAT solver.
All of these combinations are irrelevant, because the only important
information would be the set of actions occuring up to $t$ to reach the goal.

\item[c)]
Follow the approach of SATPLAN, and break apart the disjuctive goal into
subgoals that are tackled as individual SAT problems within a loop. Iterate
through each subgoal, until a satisficing assignment is found.

\end{enumerate}

\item[\textbf{2.}]
Any situation where three interchangeable actions can reach a goal, and two
non-interchangeable actions can reach the same goal, is sufficient. Suppose a
person wants to move three files from their computer to some other computer on
the internet. Their OS shell allows them to take the following actions:

Move(x). Moves the file or directory x from the starting computer to the goal
computer.

MoveAllFolder(x). Moves all three of the files to some directory x.

Now there are basically two ways to reach the goal. First, the Move() action
can be executed three times, once for each individual file. This is a single
layer of a layered plan, if the natural assumption is made that moving one
file does affect the ability to move the other two.

Second, MoveAllFolder(f) can move all of the files into a single folder f,
which can then be moved using Move(f). This is a two layer plan, because the
ability to reach the goal using Move(f) is contingent upon the first action.
Order matters here.

The first plan is layer-optimal, while the second plan is action-optimal.

\item[\textbf{3.}]
\begin{enumerate}
\item[F-F1]
Here at(r1,l2,0) can be a true fluent, so that no move is required to get
to produce at(r1,l2,1). This is not legal because this is inconsistent
with the initial state and violates condition (a).
\item[F-F2]
Here at(r1,l1,1) can be a true fluent, so that no move is made from the initial
state of \{at(r1,l1,0) $\vee \neg$at(r1,l2,0)\}. This is not legal because this
is inconsistent with the goal state and violates condition (b).
\item[F-F3]
Here move(r1,l1,l2,0) and move(r1,l2,l1,0) can both be true, along with the
initial and goal fluents. This is not legal because if both of these actions
are taken in the given order, the robot will move back to where it started
instead of moving to l2, which is presupposed by setting the goal fluent to
true. Therefore, condition (c) is violated.
\item[F-F4]
Here all of the move fluents can be false, while the initial and goal fluents
are true. This is not legal, because the robot simply jumps from the initial
state to the goal state without taking an action, violating condition (c).
\end{enumerate}

\item[\textbf{4.}]
The distance to an optimal relaxed solution is an admissible heuristic for the
distance to an optimal (or non-optimal) non-relaxed plan, because a relaxed
problem has constraints that are a subset of the constraints of a non-relaxed
problem. Since there could be less constraints to satisfy in a relaxed problem,
its optimal solution could only be shorter (and never longer) than a solution
to the corresponding non-relaxed problem.

A heuristic is admissible if it never overestimates the distance to a goal,
so the distance to the optimal relaxed solution is an admissible heuristic for
the distance to any non-relaxed solution.

\item[\textbf{5.}]
(a). As an example, the set of actions will include something like the
following:

MoveNorth(1,1):\\
PRE: at(1,1)\\
ADD: at(2,1)\\
DEL: at(1,1)

In other words, each legal move will alter the state of the robot by adding its
new location fluent to the state, and removing the old one. In the relaxed
problem, previous locations will not be removed as it
traverses the grid, and it will "occupy" its current location as well as all of
its previous locations simultaneously. This is ok, because if the robot follows
the heuristic its distance to the goal will decrease with each action.
Since an action will never take it further from the goal, its previous squares
will never be closer to the goal than its current square, and its simultaneous
occupation of all of those squares will be irrelevant. I.e., the distance to
the goal in the relaxed problem will always be equal to the distance to the
goal in the non-relaxed problem.

\end{enumerate}
\end{document}
