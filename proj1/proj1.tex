\documentclass{article}
\usepackage[pdftex]{graphicx}
\usepackage{amsmath}
\usepackage{verbatim}
\usepackage{fancyvrb}
\author{Michael Anderson, Tyler McClung}
\title{Mini-Project1}
\begin{document}
\maketitle
\center{CS533}
\center{Prof. Fern}\\
\flushleft
\newpage

\section{Playing with Blocks}

\begin{enumerate}
\item[\textbf{1.}]

\begin{small}
\begin{verbatim}
Block World A:
    Put(A,B):
        Pre: {clear(B), holding(A)}
        Add: {on(A,B), clear(A), clawEmpty}
        Del: {clear(B), holding(A)}

    Pickup(A):
        Pre: {clear(A), onTable(A), clawEmpty}
        Add: {holding(A)}
        Del: {clawEmpty, onTable(A)}

    Unstack(A,B):
        Pre: {clear(A), on(A,B), clawEmpty}
        Add: {holding(A), clear(B)}
        Del: {on(A,B)}
    Drop(A):
        Pre: {holding(A)}
        Add: {clawEmpty, onTable(A)}
        Del: {holding(A)}

Block World B:
    Put(A,B):
        Pre: {clear(B), holding(A)}
        Add: {clawEmpty}
        Del: {clear(B), holding(A)}

    Insert(A,B,C):
        Pre: {on(C,B), holding(A)}
        Add: {on(C,A), on(A,B), clawEmpty}
        Del: {on(C,B), holding(A)}

    Pickup(A):
        Pre: {clear(A), clawEmpty}
        Add: {holding(A)}
        Del: {clawEmpty}

    Unstack(A,B):
        Pre: {clear(A), on(A,B), clawEmpty}
        Add: {holding(A), clear(B)}
        Del: {on(A,B)}

    Drop(A):
        Pre: {holding(A)}
        Add: {clear(A), clawEmpty}
        Del: {holding(A)}
\end{verbatim}
\end{small}

The only difference between the two definitions is that Block World B
has an insert action which allows held blocks to be directly inserted
between two objects stacked on top of one another.

Block World B's insert action will allow for shorter solutions, but
BlackBox (our choice of planners) will have to consider possible
insert actions in every state. This will create search trees that are
shallower, but also broader, than Block World A's.

\item[\textbf{2.}]
\emph{Easy:}
None of the blocks are stacked on top of one another, which seems easy.
To solve this problem, all the blocks must simply be stacked into a
single tower in the right order from the initial position.

\emph{Medium:}
There are two towers that must be merged.  In Block World A,
both towers must be destroyed in order to construct the new
one.  In BlockWorldB, the blocks of one tower can be placed
directly into the other.

\emph{Hard:}
In this problem, the blocks are positioned in such a way that every 
block must be moved at least once in order to reach the goal state.
Normally, the more steps a problem requires to solve, the harder it is
for Blackbox to find the solution quickly.

\item[\textbf{3.}]
\underline{Block World A}\\
\emph{Easy:} 1.33 seconds, 18 actions\\
\emph{Medium:} 0.7 seconds, 24 actions\\
\emph{Hard:} Failure. 10 minute 31 seconds, 28 steps

The problem that we considered ‘Medium’ was solved faster by Blackbox.
This happened because Blackbox had to explore fewer actions, since many
of the potential actions did not meet their preconditions (blocks being
in a tower limit the number of possible actions).

\underline{Block World B}\\
\emph{Easy:} Failure\\
\emph{Medium:} 1.71 seconds, 8 actions\\
\emph{Hard:} Failure

It is important to note that while the solution lengths for most puzzles
are shorter for the the second block world, the number of actions per
iteration is much greater.  This added breadth drastically increases the
search time for longer solutions.
\end{enumerate}

\section{Create your own planning domain}
\begin{enumerate}

\item[\textbf{1.}]
Our traffic jam domain is based on the board game Rush Hour,
http://eslus.com/Gizmos/rushour/rushHour.html. The object of the game is
to move a goal vehicle to a specified edge of a 6 by 6 grid to
leave a jammed parking lot. Several other obstacle vehicles block the
path. Vehicles may only move vertically or horizontally in a straight
line, and may take up either 2 or 3 squares on the grid.

\item[\textbf{2.}]

\begin{small}
\begin{verbatim}
MoveRight2H(v, s1, s2, d):
    PRE: {VEHICLE_2H(v), at_2(v, s1, s2), empty(d), next_to_right(s2, d)}
    ADD: {empty(s1), at_2(v, s2, d)}
    DEL: {empty(d), at_2(v, s1, s2)}

MoveLeft2H(v, s1, s2, d):
    PRE: {VEHICLE_2H(v), at_2(v, s1, s2), empty(d), next_to_right(d, s1)
    ADD: {empty(s2), at_2(v, d, s1)}
    DEL: {empty(d), at_2(v, s1, s2)}

MoveRight3H (v, s1, s2, s3, d):
    PRE: {VEHICLE_3H(v),at_3(v, s1, s2, s3),  empty(d), next_to_right(s3, d)}
    ADD: {empty(s1), at_3(v,s2, s3, d)}
    DEL: {empty(d), at_3(v, s1, s2, s3)}

MoveLeft3H(v, s1, s2, s3, d):
    PRE: {VEHICLE_3H(v), at_3(v, s1, s2, s3), empty(d), next_to_right(d, s1)}
    ADD: {empty(s3), at_3(v, d, s1, s2)}
    DEL: {empty(d), at_3(v, s1, s2, s3)}

MoveUp2V(v, s1, s2, d):
    PRE: {VEHICLE_2V(v), at_2(v, s1, s2), empty(d), next_to_up(s1, d)}
    ADD: {empty(s2), at_2(v, d, s1)}
    DEL: {empty(d), at_2(v, s1, s2)}

MoveDown2V(v, s1, s2, d):
    PRE: {VEHICLE_2V(v), at_2(v, s1, s2), empty(d), next_to_up(d, s2)}
    ADD: {empty(s1), at_2(v, s2, d)}
    DEL: {empty(d), at_2(v, s1, s2)}

MoveUp3V(v, s1, s2, s3, d):
    PRE: {VEHICLE_3V(v), at_3(v, s1, s2, s3), empty(d), next_to_up(s1, d)}
    ADD: {empty(s3), at_3(v, d, s1, s2)}
    DEL: {empty(d), at_3(v, s1, s2, s3)}

MoveDown3V(v, s1, s2, s3, d):
    PRE: {VEHICLE_3V(v), at_3(v, s1, s2, s3), empty(d), next_to_up(d, s3)}
    ADD: {empty(s1), at_3(v, s2, s3, d) }
    DEL: {empty(d), at_3(v, s1, s2, s3)}
\end{verbatim}
\end{small}

Each action corresponds to the movement of a vehicle object in the grid.
In order to code this domain into STRIPS, we defined up and down
movement for vehicles placed vertically on the grid, and left and right
actions for vehicles placed horizontally on the grid. Additionally,
separate actions needed to be defined for vehicles that took up 2
squares and vehicles that took up 3 squares.

\newpage

\item[\textbf{3.}]
Our easiest test problem was the following:

\begin{small}
\begin{verbatim}
1 1 - - - 2
3 - - 4 - 2
3 0 0 4 - 2
3 - - 4 - -
5 - - - 6 6
5 - 7 7 7 -
\end{verbatim}
\end{small}

Where $vehicle_1$ is represented by the 1's in the grid, $vehicle_2$ is
represented by the 2's, etc. The goal is to get $vehicle_0$ to the right
side of the grid, where the ``exit" is. In this particular initial
position it is blocked by vehicles 4 and 2.

Included with our submission is the python script proj1.py, which
converts such a grid into the corresponding set of PDDL propositions
that can be used as the facts file for BlackCox.

The solution, as given by BlackBox, to this problem is:

\begin{small}
\begin{verbatim}
----------------------------------------------------
Begin plan
1 (moveleft2h v6 sq-4-4 sq-4-5 sq-4-3)
1 (moveright2h v1 sq-0-0 sq-0-1 sq-0-2)
1 (moveleft3h v7 sq-5-2 sq-5-3 sq-5-4 sq-5-1)
1 (movedown3v v2 sq-0-5 sq-1-5 sq-2-5 sq-3-5)
2 (movedown3v v2 sq-1-5 sq-2-5 sq-3-5 sq-4-5)
2 (moveleft2h v6 sq-4-3 sq-4-4 sq-4-2)
2 (moveup3v v3 sq-1-0 sq-2-0 sq-3-0 sq-0-0)
3 (movedown3v v2 sq-2-5 sq-3-5 sq-4-5 sq-5-5)
3 (moveleft2h v6 sq-4-2 sq-4-3 sq-4-1)
3 (moveup2v v5 sq-4-0 sq-5-0 sq-3-0)
4 (moveleft3h v7 sq-5-1 sq-5-2 sq-5-3 sq-5-0)
4 (movedown3v v4 sq-1-3 sq-2-3 sq-3-3 sq-4-3)
5 (movedown3v v4 sq-2-3 sq-3-3 sq-4-3 sq-5-3)
6 (moveright2h v0 sq-2-1 sq-2-2 sq-2-3)
7 (moveright2h v0 sq-2-2 sq-2-3 sq-2-4)
8 (moveright2h v0 sq-2-3 sq-2-4 sq-2-5)
End plan
----------------------------------------------------

16 total actions in plan
0 entries in hash table, 
7 total set-creation steps (entries + hits + plan length - 1)
16 actions tried
\end{verbatim}
\end{small}

This solution can be checked to ensure its accuracy. For example, the
first action is to move the horizontally-placed $vehicle_6$, currently
occupying squares (4,4) and (4,5) left one square to (4,3). According to
the action schema, square (4,5) will be deleted from the at() predicate
associated with v6, and will be replaced by square (4,3).

\item[\textbf{4.}]
The website in section (1.) classified three of the below problems as
easy, medium, and hard. We found the insane problem on a website
claiming to have the hardest problems possible in this domain,
http://cs.ulb.ac.be/~fservais/rushhour/index.php?. The
site claims that it requires at least 93 moves to solve.

\emph{Easy:} 0.08 seconds, 16 actions\\
\emph{Medium:} 0.26 seconds, 23 actions\\
\emph{Hard:} Failure\\
\emph{Insane:} Failure

After running the planner on four sample problems of varying
difficulties (from easy to the most challenging), it was clear that
Blackbox performs well only in relatively shallow, narrow search
spaces. Since Blackbox is essentially an iterative deepening descent
search, it must iterate the entire search tree of depth n, where n is
the number of steps of the optimal solution.  Deep, wide trees, that
grow exponentially per iteration, cause Blackbox to fail for problems
with significant large search spaces.

The solutions that Blackbox provides are quite useful. As expected,
every action in a layer must be executed before actions in the next
layer, but are order-independent of actions in the same layer. To solve
a puzzle, the actions in BlackBox's solution can be executed 
top-to-bottom to reach the goal state. All of the solutions have the
added benefit of being optimal in the number of actions given.

\end{enumerate}
\end{document}
